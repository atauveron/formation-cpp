% -*- ispell-dictionary: "french" -*-

\documentclass{beamer}
\usetheme{metropolis}

\usepackage[french]{babel}
\usepackage[utf8]{inputenc}
\usepackage[T1]{fontenc}

% Listes sur 2 colonnes
\usepackage{multicol}

% Tableaux
\usepackage{array}
\usepackage{multirow}

% Images, couleurs
\usepackage{graphicx}
\usepackage{xcolor}

% Mathématiques
\usepackage{mathtools,amssymb,stmaryrd}
\numberwithin{equation}{section}
% Suite
\newcommand{\bs}[1]{\boldsymbol{#1}}
\newcommand{\dtemp}{\frac{d}{dt}}
\newcommand{\dpart}[2][]{\frac{\partial #1}{\partial #2}}
\newcommand{\inv}[1]{\frac{1}{#1}}
\newcommand{\card}[1]{\text{card}\,{#1}}

% Code
\usepackage{listings}

% Références
\usepackage{hyperref}

% Titre
\title{Formation \texttt{C++}}
\date{Septembre 2018}
\author[A.~\textsc{Tauveron}]{Aimery~\textsc{Tauveron-\,-Jalenques}}
\institute{\href{https://viarezo.fr/}{ViaRézo}}



\begin{document}

% Configuration pour listings
\lstset{frame=single, frameround=tttt, rulecolor=\color{blue}}
\lstset{breaklines=true, showstringspaces=false}
\lstset{language=[11]C++}

\frame{\titlepage}

% Logo VR
\addtobeamertemplate{frametitle}{}{
  \begin{tikzpicture}[remember picture,overlay]
    \node[anchor=north east,yshift=0pt] at (current page.north east) {\includegraphics[height=0.75cm]{VR_vertical.png}};
  \end{tikzpicture}
}

\section*{Introduction}
\label{sec:introduction}
\begin{frame}{Introduction}
  \begin{quotation}
    C++ est un langage de programmation compilé permettant la programmation sous de multiples paradigmes (comme la programmation procédurale, orientée objet ou générique). [\dots{}]

    Créé initialement par Bjarne Stroustrup dans les années 1980, le langage C++ est aujourd'hui normalisé par l'ISO. Sa première normalisation date de 1998, ensuite amendée par l'erratum technique de 2003. Une importante mise à jour a été ratifiée et publiée par l'ISO en septembre 2011 sous le nom de C++11. Depuis, des mises à jour sont publiées régulièrement : en 2014 puis en 2017.
  \end{quotation}

  \flushright{Extrait adapté de \href{https://fr.wikipedia.org/wiki/C\%2B\%2B}{Wikipédia}}
\end{frame}

\begin{frame}{Introduction}
  Ce n'est pas :
  \begin{itemize}
    \item une formation C (pas de \texttt{(int*)malloc(...)}),
    \item une formation POO,
    \item un tutoriel.
  \end{itemize}
\end{frame}


\frame{\tableofcontents}


\section{\texttt{C++} 101}
\label{sec:cpp101}
% -*- ispell-dictionary: "french" -*-

\begin{frame}[fragile]{Hello world!}
  \begin{lstlisting}
    #include <iostream>
    
    int main () {
      std::cout << "Hello world!\n";
      return 0;
    }
  \end{lstlisting}
\end{frame}

\begin{frame}[fragile]{Compilation}
  Il faut compiler !
  \begin{itemize}
    \item GCC, Clang, Microsoft Visual C++
  \end{itemize}

  \begin{lstlisting}[language=bash]
    g++ main.cpp
  \end{lstlisting}

  On peut exécuter :
  \begin{lstlisting}
    ./a.out
    Hello world!
  \end{lstlisting}
\end{frame}

\begin{frame}[fragile]{Compilation}
  \begin{lstlisting}[language=bash]
    g++ -O3 -Wall main.cpp -o hello
  \end{lstlisting}

  C'est mieux !
  \begin{lstlisting}
    ./hello
    Hello world!
  \end{lstlisting}
\end{frame}

\begin{frame}[fragile]{Définition de variables}
  \begin{lstlisting}
    #include <iostream>

    int main () {
      int a;
      int b (0);
      int c = 0;
      std::cout << "a is: ";
      std::cin >> a;
      b = 2*a;
      std::cout << "a is: " << a << "\n";
      std::cout << "b is: " << b << "\n";
      std::cout << "2*a is: " << 2*a << "\n";
      return 0;
    }
  \end{lstlisting}
\end{frame}

\begin{frame}{Les variables}
  Des types :
  \begin{itemize}
  \item numériques : \texttt{int} (\texttt{unsigned}, \texttt{long}, \texttt{short}), \texttt{float}, \texttt{double}
  \item non-numériques : \texttt{char}, \texttt{bool}
  \item \texttt{void}
  \item autres (objets\dots{})
  \end{itemize}

  Trois catégories de variables :
  \begin{itemize}
  \item variables : \textit{type} \textit{var}
  \item références : \textit{type} \texttt{\&}\textit{var}
  \item pointeurs : \textit{type} \texttt{*}\textit{var}
  \end{itemize}

  Un qualificateur : \texttt{const}
\end{frame}

\begin{frame}[fragile]{Les variables}
  \begin{lstlisting}
    int main () {
      int a (2);
      int b (a);
      int &c (a);
      int *d = a;
      std::cout << "a is: " << a << "\n";
      std::cout << "b is: " << b << "\n";
      std::cout << "c is: " << c << "\n";
      std::cout << "d is: " << d << "\n";
      return 0;
    }
  \end{lstlisting}
\end{frame}

\begin{frame}[fragile]{Les variables}
  \begin{lstlisting}
    int main () {
      int a (2);
      int b (a);
      int &c (a);
      int *d = &a; // l'adresse memoire de a
      std::cout << "a is: " << a << "\n";
      std::cout << "b is: " << b << "\n";
      std::cout << "c is: " << c << "\n";
      std::cout << "d is: " << d << "\n";
      return 0;
    }
  \end{lstlisting}
\end{frame}

\begin{frame}[fragile]{Les variables}
  \begin{lstlisting}
    int main () {
      int a (2);
      int b (a);
      int &c (a);
      int *d = &a; // l'adresse memoire de a
      std::cout << "a is: " << a << "\n";
      std::cout << "b is: " << b << "\n";
      std::cout << "c is: " << c << "\n";
      std::cout << "d is: " << *d << "\n"; /* la valeur pointee par d */
      return 0;
    }
  \end{lstlisting}
\end{frame}

\begin{frame}[fragile]{Les tableaux}
  \begin{lstlisting}
    int a [8];
    char w [256];
    a[0] = 0;
    a[1] = 4;
    std::cout << a[0];
  \end{lstlisting}

  \begin{itemize}
  \item Hérités de C
  \item Indexés de 0 à $n-1$
  \item Des pointeurs $\implies$ pas de vérification des indices
  \end{itemize}

  \begin{lstlisting}
    int *b = a; // Correct
    a[11] = 32; // Undefined behaviour, hi sha
  \end{lstlisting}  
\end{frame}

\begin{frame}[fragile]{Les fonctions}
  Définition :
  \begin{lstlisting}[escapechar=ù]
    ù\textit{return\_type}ù ù\textit{function\_name}ù (ù\textit{arguments}ù) {
      // code
      return ù\textit{return\_value}ù
    }
  \end{lstlisting}

  Appel :
  \begin{lstlisting}[escapechar=ù]
    ù\textit{variable}ù = ù\textit{function\_name}ù (ù\textit{arguments}ù);
  \end{lstlisting}

  Surcharge de fonctions possible
  
\end{frame}

\begin{frame}[fragile]{La fonction \texttt{main}}
  \begin{lstlisting}
    int main (int argc, char **argv) {
      // code
      return 0;
    }
  \end{lstlisting}
 ou bien
  \begin{lstlisting}
    int main (int argc, char *argv[]) {
      // code
      return 0;
    }
  \end{lstlisting}
\end{frame}

\begin{frame}{Des opérateurs}
  Dans le désordre :
  \begin{itemize}
  \item arithmétiques : +, -, *, /, \%
  \item de comparaison : ==, !=, <, >, <=, >=
  \item logiques : \&\&, ||, !
  \item binaires : \&, |, \^{}, \~{}
  \item mémoire : \&, *
  \end{itemize}

  En \texttt{C++}, le contexte est important !
\end{frame}

\begin{frame}[fragile]{Les \texttt{namespace}s}
  Pourquoi \texttt{std::cout} et non \texttt{cout} ? \texttt{int} et non \texttt{std::int} ?
  
  \textit{int} est dans le \texttt{namespace} global, \texttt{cout} dans le \texttt{namespace} \texttt{std}.
  
  \texttt{std} contient la bibliothèque standard.

  On peut utiliser :
  \begin{lstlisting}
    using namespace std;
    
    int main () {
      cout << "Hello world!\n";
      return 0;
    }
  \end{lstlisting}
\end{frame}

\begin{frame}[fragile]{Les \texttt{namespace}s}
  \begin{lstlisting}
    namespace foo {
      bool var (true);
    }

    namespace bar {
      double var (1.0);
    }
    int main () {
      std::cout << "foo::var is:" << foo:var << "\n";
      std::cout << "bar::var is:" << bar:var << "\n";
      return 0;
    }
  \end{lstlisting}
\end{frame}



\section{Programmation orientée objet en \texttt{C++}}
\label{sec:poo}
% -*- ispell-dictionary: "french" -*-

\begin{frame}[fragile]{Structures et objets}
  Définir de nouveau types
  \begin{lstlisting}
    struct atom {
      int atomic; // Atomic number
      int mass; // Mass number
    };

    atom carbon12;
    carbon12.atomic = 6;
    carbon12.mass = 12;
  \end{lstlisting}

  Membres d'une fonction : variables (attributs), fonctions (méthodes)
\end{frame}

\begin{frame}[fragile]{Structures et objets}
  \begin{lstlisting}
    class Atom {
    private:
      int atomic;
      int mass;
    public:
      Atom(int a, int m) {
        atomic = a; mass = m;
      }
    };

    atom carbon12 (6,12);
    carbon12.mass = 13; // Forbidden!
  \end{lstlisting}
\end{frame}

\begin{frame}{Sructures et objets}
  \textcolor{red}{Les attributs d'une classe sont toujours \textbf{privés}.}

  Les accès se font donc grâce aux méthodes de la classe.
\end{frame}

\begin{frame}[fragile]{Un objet minimaliste}
  \begin{lstlisting}
    class A {
    private:
      // attributes
      // private methods
    public:
      A(...) {...} // constructor
      ~A(...) {...} // destructor
      A(A &a) {...} // copy constructor
      // public methods
    };
  \end{lstlisting}
\end{frame}


\section{Le \texttt{C++} moderne}


\section{Introduction à la librairie standard}


\section*{Conclusion}
\label{sec:conclusion}
\begin{frame}{Ressources}
  \begin{itemize}
    \item \href{https://stackoverflow.com/}{Stack Overflow}
    \item \href{http://www.cplusplus.com/}{cplusplus.com}
  \end{itemize}
\end{frame}

\end{document}