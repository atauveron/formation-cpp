% -*- ispell-dictionary: "french" -*-

\begin{frame}[fragile]{Hello world!}
  \begin{lstlisting}
#include <iostream>

int main () {
  std::cout << "Hello world!\n";
  return 0;
}
  \end{lstlisting}
\end{frame}

\begin{frame}[fragile]{Compilation}
  Il faut compiler !
  \begin{itemize}
    \item GCC, Clang, Microsoft Visual C++
  \end{itemize}

  \begin{lstlisting}[language=bash]
g++ main.cpp
  \end{lstlisting}

  On peut exécuter :
  \begin{lstlisting}
./a.out
Hello world!
  \end{lstlisting}
\end{frame}

\begin{frame}[fragile]{Compilation}
  \begin{lstlisting}[language=bash]
g++ -O3 -Wall main.cpp -o hello
  \end{lstlisting}

  C'est mieux !
  \begin{lstlisting}
./hello
Hello world!
  \end{lstlisting}
\end{frame}

\begin{frame}[fragile]{Définition de variables}
  \begin{lstlisting}
#include <iostream>

int main () {
  int a;
  int b (0);
  int c = 0;
  std::cout << "a is: ";
  std::cin >> a;
  b = 2*a;
  std::cout << "a is: " << a << "\n";
  std::cout << "b is: " << b << "\n";
  std::cout << "2*a is: " << 2*a << "\n";
  return 0;
}
  \end{lstlisting}
\end{frame}

\begin{frame}{Les variables}
  Des types :
  \begin{itemize}
  \item numériques : \texttt{int} (\texttt{unsigned}, \texttt{long}, \texttt{short}), \texttt{float}, \texttt{double}
  \item non-numériques : \texttt{char}, \texttt{bool}
  \item \texttt{void}
  \item autres (objets\dots{})
  \end{itemize}

  Trois catégories de variables :
  \begin{itemize}
  \item variables : \textit{type} \textit{var}
  \item références : \textit{type} \texttt{\&}\textit{var}
  \item pointeurs : \textit{type} \texttt{*}\textit{var}
  \end{itemize}

  Un qualificateur : \texttt{const}
\end{frame}

\begin{frame}[fragile]{Les variables statiques}
  \begin{lstlisting}
int main () {
  int a (2);
  int b (a);
  int &c (a);
  int *d = a;
  std::cout << "a is: " << a << "\n";
  std::cout << "b is: " << b << "\n";
  std::cout << "c is: " << c << "\n";
  std::cout << "d is: " << d << "\n";
  return 0;
}
  \end{lstlisting}
\end{frame}

\begin{frame}[fragile]{Les variables statiques}
  \begin{lstlisting}
    int main () {
      int a (2);
      int b (a);
      int &c (a);
      int *d = &a; // address of a
      std::cout << "a is: " << a << "\n";
      std::cout << "b is: " << b << "\n";
      std::cout << "c is: " << c << "\n";
      std::cout << "d is: " << d << "\n";
      return 0;
    }
  \end{lstlisting}
\end{frame}

\begin{frame}[fragile]{Les variables statiques}
  \begin{lstlisting}
int main () {
  int a (2);
  int b (a);
  int &c (a);
  int *d = &a; // address of a
  std::cout << "a is: " << a << "\n";
  std::cout << "b is: " << b << "\n";
  std::cout << "c is: " << c << "\n";
  std::cout << "d is: " << *d << "\n"; /* the value pointed to by d */
  return 0;
}
  \end{lstlisting}
\end{frame}

\begin{frame}[fragile]{Les tableaux}
  \begin{lstlisting}
int a [8];
char w [256];
a[0] = 0;
a[1] = 4;
std::cout << a[0];
  \end{lstlisting}

  \begin{itemize}
  \item Hérités de C
  \item Indexés de 0 à $n-1$
  \item Des pointeurs $\implies$ pas de vérification des indices
  \end{itemize}

  \begin{lstlisting}
int *b = a; // Correct
a[11] = 32; // Undefined behaviour
  \end{lstlisting}  
\end{frame}

\begin{frame}[fragile]{Cycle de vie des variables statiques}
  \begin{itemize}
  \item Création à la déclaration
  \item Destruction à la fin du bloc \texttt{\{\dots\}}
  \end{itemize}

  \begin{lstlisting}
int main () {
  // maybe some code
  int a (0);
  unsigned int b;
  {
    double c (.5);
  }
  // c doesn't exist here
  return 0;
}
  \end{lstlisting}
\end{frame}

\begin{frame}[fragile]{Les variables dynamiques}
  Deux couples d'opérateurs : \texttt{new} et \texttt{delete}, \texttt{malloc} et \texttt{free}
  \begin{lstlisting}
int *a = new int(2);
delete a;
a = nullptr;
  \end{lstlisting}

  Pour les tableaux : \texttt{new[]} et \texttt{delete[]}
  \begin{lstlisting}
int *a = new int[4];
delete[] a;
a = nullptr;
  \end{lstlisting}
\end{frame}

\begin{frame}[fragile]{Les variables dynamiques}
  \begin{lstlisting}
int main () {
  int *a;
  {
    int *b = new int(4);
    a = b;
  }
  // b doesn't exist here
  // but a does
  std::cout << "a is: " << *a << "\n";
  delete a;
  a = nullptr;
  return 0;
}
  \end{lstlisting}
\end{frame}

\begin{frame}[fragile]{Les structures de contrôle}
  \begin{lstlisting}[escapechar=ù]
if ( ù\textit{condition}ù ) {
  // code
} else if ( ù\textit{condition}ù ) {
  // code
} else {
  // code
}
  \end{lstlisting}

  \begin{lstlisting}[escapechar=ù]
if ( ù\textit{condition}ù )
// single line of code
else
// single line of code
  \end{lstlisting}
\end{frame}

\begin{frame}[fragile]{Les structures de contrôle}
  \begin{lstlisting}[escapechar=ù]
while ( ù\textit{condition}ù ) {
  // code
}
  \end{lstlisting}
\end{frame}

\begin{frame}[fragile]{Les structures de contrôle}
  \begin{lstlisting}[escapechar=ù]
for ( ù\textit{initialisation}ù ; ù\textit{condition}ù ; ù\textit{increase}ù ) {
  // code
}
  \end{lstlisting}

  \begin{lstlisting}
for ( int i=0 ; i<10 ; ++i ) {
  // code
}
  \end{lstlisting}
  \begin{lstlisting}
for ( n=0, i=100 ; n!=i ; ++n, --i ) {
  // code
}
  \end{lstlisting}
\end{frame}

\begin{frame}[fragile]{Les structures de contrôle}
  \begin{lstlisting}[escapechar=ù]
switch ( ù\textit{expression}ù ) {
  case ù\textit{constant}ù:
  // code
  break;
  case ù\textit{constant}ù:
  // code
  break;
  default:
  //code
}
  \end{lstlisting}
\end{frame}

\begin{frame}[fragile]{Les fonctions}
  Définition :
  \begin{lstlisting}[escapechar=ù]
ù\textit{return\_type}ù ù\textit{function\_name}ù (ù\textit{arguments}ù) {
  // code
  return ù\textit{return\_value}ù;
}
  \end{lstlisting}

  Appel :
  \begin{lstlisting}[escapechar=ù]
ù\textit{variable}ù = ù\textit{function\_name}ù (ù\textit{arguments}ù);
  \end{lstlisting}

  Surcharge de fonctions possible
\end{frame}

\begin{frame}[fragile]{La fonction \texttt{main}}
  \begin{lstlisting}
int main (int argc, char **argv) {
  // code
  return 0;
}
  \end{lstlisting}
 ou bien
  \begin{lstlisting}
int main (int argc, char *argv[]) {
  // code
  return 0;
}
  \end{lstlisting}
\end{frame}

\begin{frame}{Des opérateurs}
  Dans le désordre :
  \begin{itemize}
  \item arithmétiques : +, -, *, /, \%
  \item de comparaison : ==, !=, <, >, <=, >=
  \item logiques : \&\&, ||, !
  \item binaires : \&, |, \^{}, \~{}
  \item mémoire : \&, *
  \end{itemize}

  En \texttt{C++}, le contexte est important !
\end{frame}

\begin{frame}[fragile]{Les \texttt{namespace}s}
  Pourquoi \texttt{std::cout} et non \texttt{cout} ? \texttt{int} et non \texttt{std::int} ?
  
  \textit{int} est dans le \texttt{namespace} global, \texttt{cout} dans le \texttt{namespace} \texttt{std}.
  
  \texttt{std} contient la bibliothèque standard.

  On peut utiliser :
  \begin{lstlisting}
using namespace std;

int main () {
  cout << "Hello world!\n";
  return 0;
}
  \end{lstlisting}
\end{frame}

\begin{frame}[fragile]{Les \texttt{namespace}s}
  \begin{lstlisting}
namespace foo {
  bool var (true);
}
namespace bar {
  double var (1.0);
}
int main () {
  std::cout << "foo::var is:" << foo::var << "\n";
  std::cout << "bar::var is:" << bar::var << "\n";
  return 0;
}
  \end{lstlisting}
\end{frame}
