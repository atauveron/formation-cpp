% -*- ispell-dictionary: "french" -*-

\begin{frame}[fragile]{Avant d'aller plus loin}
  Pour activer les optimisations à la compilation :
  \begin{lstlisting}[language=bash]
g++ -Ox ...
  \end{lstlisting}

  \begin{itemize}
    \item 0 : aucune optimisation
    \item 1 à 3 : optimisation de plus en plus agressive de la vitesse
    \item \texttt{s} : optimisation de la taille de l'exécutable
  \end{itemize}
\end{frame}

\begin{frame}{Les macros}
  Mais qu'est réellement l'instruction \texttt{\#include <...>} ?
\end{frame}

\begin{frame}[fragile]{Les macros}
  \begin{itemize}
    \item Mécanisme d'exécution d'instructions à la compilation
    \item Hérité de C
  \end{itemize}

  \begin{lstlisting}
#include <cmath>

#define PI 3.1415
  \end{lstlisting}
\end{frame}

\begin{frame}[fragile]{Le mot clé \texttt{constexpr}}
  \texttt{constexpr} sert à indiquer qu'une expression est évaluable à la compilation.

  \begin{lstlisting}
constexpr double pi = 3.1415;
  \end{lstlisting}

  Il implique le qualificatif \texttt{const} pour les variables.  
\end{frame}

\begin{frame}[fragile]{Le mot clé \texttt{constexpr}}
  \begin{lstlisting}
constexpr std::vector<double> axis_x = {1, 0, 0};

constexpr int x = 1;
constexpr int y = 3;
constexpr int z = x + 2 * y;
  \end{lstlisting}
\end{frame}

\begin{frame}[fragile]{Le mot clé \texttt{constexpr}}
  \begin{lstlisting}
constexpr int factorial(int n) {
  return n <= 1 ? 1 : (n * factorial(n - 1));
}
  \end{lstlisting}
\end{frame}

\begin{frame}[fragile]{\texttt{constexpr} pour les classes}
  Il faut au moins un constructeur \texttt{constexpr}.

  \begin{lstlisting}
class GaussInt {
  public:
    constexpr GaussInt (int a=0, int b=0): m_real(a), m_imag(b) {}
    constexpr double norm () const { return std::sqrt(m_real*m_real + m_imag*m_imag); }
    constexpr GaussInt operator+(const GaussInt &other) const {
      return GaussInt(this->m_real + other.m_real, this->m_imag + other.m_imag);
    }
  };
  \end{lstlisting}
\end{frame}

\begin{frame}[fragile]{\texttt{constexpr} pour les classes}
  \begin{lstlisting}
constexpr GaussInt a;
constexpr GaussInt b (2,2);
constexpr GaussInt c (2,1);
constexpr GaussInt d (b + c);
constexpr double a_norm =  a.norm();
std::cout << "Norm of a: " << a_norm << std::endl;
std::cout << "Norm of d: " << d.norm() << std::endl;
constexpr auto fac = factorial(5);
std::cout << "Factorial of 5 is " << fac << std::endl;
  \end{lstlisting}
\end{frame}
