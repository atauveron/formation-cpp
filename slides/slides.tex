% -*- ispell-dictionary: "french" -*-

\documentclass{beamer}
\usetheme{metropolis}

\usepackage[french]{babel}
\usepackage[T1]{fontenc}
\usepackage[utf8]{inputenc}

% Images, couleurs
\usepackage{graphicx}
\usepackage{xcolor}

% Code
\usepackage{listings}

% Références
\usepackage{hyperref}

% Schémas
\usepackage{tikz}
\usetikzlibrary{positioning}

% Titre
\title{Formation \texttt{C++}}
\date{Septembre 2018}
\author[A.~\textsc{Tauveron}]{Aimery~\textsc{Tauveron-\,-Jalenques}}
\institute{\href{https://viarezo.fr/}{ViaRézo}}


\begin{document}

% Configuration pour listings
\lstset{frame=single, frameround=tttt, rulecolor=\color{blue}}
\lstset{breaklines=true, showstringspaces=false}
\lstset{language=[11]C++}

\frame{\titlepage}

% Logo VR
\addtobeamertemplate{frametitle}{}{
  \begin{tikzpicture}[remember picture,overlay]
    \node[anchor=north east,yshift=0pt] at (current page.north east) {\includegraphics[height=0.75cm]{VR_vertical.png}};
  \end{tikzpicture}
}

\section*{Introduction}
\label{sec:introduction}
\begin{frame}{Introduction}
  \begin{quotation}
    C++ est un langage de programmation compilé permettant la programmation sous de multiples paradigmes (comme la programmation procédurale, orientée objet ou générique). [\dots{}]

    Créé initialement par Bjarne Stroustrup dans les années 1980, le langage C++ est aujourd'hui normalisé par l'ISO. Sa première normalisation date de 1998, ensuite amendée par l'erratum technique de 2003. Une importante mise à jour a été ratifiée et publiée par l'ISO en septembre 2011 sous le nom de C++11. Depuis, des mises à jour sont publiées régulièrement : en 2014 puis en 2017.
  \end{quotation}

  \flushright{Extrait adapté de \href{https://fr.wikipedia.org/wiki/C\%2B\%2B}{Wikipédia}}
\end{frame}

\begin{frame}{Introduction}
  Ce n'est pas :
  \begin{itemize}
  \item une formation C (pas de \texttt{(int*)malloc(...)}),
  \item une formation compilation,
  \item une formation POO,
  \item un tutoriel.
  \end{itemize}
\end{frame}


\frame{\tableofcontents}


\section{\texttt{C++} 101}
\label{sec:cpp101}
% -*- ispell-dictionary: "french" -*-

\begin{frame}[fragile]{Hello world!}
  \begin{lstlisting}
#include <iostream>

int main () {
  std::cout << "Hello world!\n";
  return 0;
}
  \end{lstlisting}
\end{frame}

\begin{frame}[fragile]{Compilation}
  Il faut compiler !
  \begin{itemize}
    \item GCC, Clang, Microsoft Visual C++
  \end{itemize}

  \begin{lstlisting}[language=bash]
g++ hello.cpp
  \end{lstlisting}

  On peut exécuter :
  \begin{lstlisting}
./a.out
Hello world!
  \end{lstlisting}
\end{frame}

\begin{frame}[fragile]{Compilation}
  \begin{lstlisting}[language=bash]
g++ -Wall hello.cpp -o hello
  \end{lstlisting}

  C'est mieux !
  \begin{lstlisting}
./hello
Hello world!
  \end{lstlisting}
\end{frame}

\begin{frame}[fragile]{Définition de variables}
  \begin{lstlisting}
#include <iostream>

int main () {
  int a;
  int b (0);
  int c = 0;
  std::cout << "a is: ";
  std::cin >> a;
  b = 2*a;
  std::cout << "a is: " << a << "\n";
  std::cout << "b is: " << b << "\n";
  std::cout << "2*a is: " << 2*a << "\n";
  return 0;
}
  \end{lstlisting}
\end{frame}

\begin{frame}{Les variables}
  Des types :
  \begin{itemize}
  \item numériques : \texttt{int} (\texttt{unsigned}, \texttt{long}, \texttt{short}), \texttt{float}, \texttt{double}
  \item non-numériques : \texttt{char}, \texttt{bool}
  \item \texttt{void}
  \item autres (objets\dots{})
  \end{itemize}

  Trois catégories de variables :
  \begin{itemize}
  \item variables : \textit{type} \textit{var}
  \item références : \textit{type} \texttt{\&}\textit{var}
  \item pointeurs : \textit{type} \texttt{*}\textit{var}
  \end{itemize}

  Un qualificateur : \texttt{const}
\end{frame}

\begin{frame}[fragile]{Les variables statiques}
  \begin{lstlisting}
int main () {
  int a (2);
  int b (a);
  int &c (a);
  int *d = a;
  std::cout << "a is: " << a << "\n";
  std::cout << "b is: " << b << "\n";
  std::cout << "c is: " << c << "\n";
  std::cout << "d is: " << d << "\n";
  return 0;
}
  \end{lstlisting}
\end{frame}

\begin{frame}[fragile]{Les variables statiques}
  \begin{lstlisting}
    int main () {
      int a (2);
      int b (a);
      int &c (a);
      int *d = &a; // address of a
      std::cout << "a is: " << a << "\n";
      std::cout << "b is: " << b << "\n";
      std::cout << "c is: " << c << "\n";
      std::cout << "d is: " << d << "\n";
      return 0;
    }
  \end{lstlisting}
\end{frame}

\begin{frame}[fragile]{Les variables statiques}
  \begin{lstlisting}
int main () {
  int a (2);
  int b (a);
  int &c (a);
  int *d = &a; // address of a
  std::cout << "a is: " << a << "\n";
  std::cout << "b is: " << b << "\n";
  std::cout << "c is: " << c << "\n";
  std::cout << "d is: " << *d << "\n"; /* the value pointed to by d */
  return 0;
}
  \end{lstlisting}
\end{frame}

\begin{frame}[fragile]{Les tableaux}
  \begin{lstlisting}
int a [8];
char w [256];
a[0] = 0;
a[1] = 4;
std::cout << a[0];
  \end{lstlisting}

  \begin{itemize}
  \item Hérités de C
  \item Indexés de 0 à $n-1$
  \item Des pointeurs $\implies$ pas de vérification des indices
  \end{itemize}

  \begin{lstlisting}
int *b = a; // Correct
a[11] = 32; // Undefined behaviour
  \end{lstlisting}  
\end{frame}

\begin{frame}[fragile]{Cycle de vie des variables statiques}
  \begin{itemize}
  \item Création à la déclaration
  \item Destruction à la fin du bloc \texttt{\{\dots\}}
  \end{itemize}

  \begin{lstlisting}
int main () {
  // maybe some code
  int a (0);
  unsigned int b;
  {
    double c (.5);
  }
  // c doesn't exist here
  return 0;
}
  \end{lstlisting}
\end{frame}

\begin{frame}[fragile]{Les variables dynamiques}
  Deux couples d'opérateurs : \texttt{new} et \texttt{delete}, \texttt{malloc} et \texttt{free}
  \begin{lstlisting}
int *a = new int(2);
delete a;
a = nullptr;
  \end{lstlisting}

  Pour les tableaux : \texttt{new[]} et \texttt{delete[]}
  \begin{lstlisting}
int *a = new int[4];
delete[] a;
a = nullptr;
  \end{lstlisting}
\end{frame}

\begin{frame}[fragile]{Les variables dynamiques}
  \begin{lstlisting}
int main () {
  int *a;
  {
    int *b = new int(4);
    a = b;
  }
  // b doesn't exist here
  // but a does
  std::cout << "a is: " << *a << "\n";
  delete a;
  a = nullptr;
  return 0;
}
  \end{lstlisting}
\end{frame}

\begin{frame}[fragile]{Les structures de contrôle}
  \begin{lstlisting}[escapechar=ù]
if ( ù\textit{condition}ù ) {
  // code
} else if ( ù\textit{condition}ù ) {
  // code
} else {
  // code
}
  \end{lstlisting}

  \begin{lstlisting}[escapechar=ù]
if ( ù\textit{condition}ù )
// single line of code
else
// single line of code
  \end{lstlisting}
\end{frame}

\begin{frame}[fragile]{Les structures de contrôle}
  \begin{lstlisting}[escapechar=ù]
while ( ù\textit{condition}ù ) {
  // code
}
  \end{lstlisting}
\end{frame}

\begin{frame}[fragile]{Les structures de contrôle}
  \begin{lstlisting}[escapechar=ù]
for ( ù\textit{initialisation}ù ; ù\textit{condition}ù ; ù\textit{increase}ù ) {
  // code
}
  \end{lstlisting}

  \begin{lstlisting}
for ( int i=0 ; i<10 ; ++i ) {
  // code
}
  \end{lstlisting}
  \begin{lstlisting}
for ( n=0, i=100 ; n!=i ; ++n, --i ) {
  // code
}
  \end{lstlisting}
\end{frame}

\begin{frame}[fragile]{Les structures de contrôle}
  \begin{lstlisting}[escapechar=ù]
switch ( ù\textit{expression}ù ) {
  case ù\textit{constant}ù:
  // code
  break;
  case ù\textit{constant}ù:
  // code
  break;
  default:
  //code
}
  \end{lstlisting}
\end{frame}

\begin{frame}[fragile]{Les fonctions}
  Définition :
  \begin{lstlisting}[escapechar=ù]
ù\textit{return\_type}ù ù\textit{function\_name}ù (ù\textit{arguments}ù) {
  // code
  return ù\textit{return\_value}ù;
}
  \end{lstlisting}

  Appel :
  \begin{lstlisting}[escapechar=ù]
ù\textit{variable}ù = ù\textit{function\_name}ù (ù\textit{arguments}ù);
  \end{lstlisting}

  Surcharge de fonctions possible
\end{frame}

\begin{frame}[fragile]{Les fonctions}
  Trois manières de passer un argument :
  \begin{itemize}
  \item par valeur
    \begin{lstlisting}
int f (int a) {...}
    \end{lstlisting}
  \item par référence
    \begin{lstlisting}
int f (int &a) {...}
    \end{lstlisting}
  \item par pointeur
    \begin{lstlisting}
int f (int *a) {...}
    \end{lstlisting}
  \end{itemize}
\end{frame}

\begin{frame}[fragile]{La fonction \texttt{main}}
  \begin{lstlisting}
int main (int argc, char **argv) {
  // code
  return 0;
}
  \end{lstlisting}
 ou bien
  \begin{lstlisting}
int main (int argc, char *argv[]) {
  // code
  return 0;
}
  \end{lstlisting}
\end{frame}

\begin{frame}{Des opérateurs}
  Dans le désordre :
  \begin{itemize}
  \item arithmétiques : +, -, *, /, \%
  \item de comparaison : ==, !=, <, >, <=, >=
  \item logiques : \&\&, ||, !
  \item binaires : \&, |, \^{}, \~{}
  \item mémoire : \&, *
  \end{itemize}

  En \texttt{C++}, le contexte est important !
\end{frame}

\begin{frame}[fragile]{Les \texttt{namespace}s}
  Pourquoi \texttt{std::cout} et non \texttt{cout} ? \texttt{int} et non \texttt{std::int} ?
  
  \textit{int} est dans le \texttt{namespace} global, \texttt{cout} dans le \texttt{namespace} \texttt{std}.
  
  \texttt{std} contient la bibliothèque standard.

  On peut utiliser :
  \begin{lstlisting}
using namespace std;

int main () {
  cout << "Hello world!\n";
  return 0;
}
  \end{lstlisting}
\end{frame}

\begin{frame}[fragile]{Les \texttt{namespace}s}
  \begin{lstlisting}
namespace foo {
  bool var (true);
}
namespace bar {
  double var (1.0);
}
int main () {
  std::cout << "foo::var is:" << foo::var << "\n";
  std::cout << "bar::var is:" << bar::var << "\n";
  return 0;
}
  \end{lstlisting}
\end{frame}



\section{Programmation orientée objet en \texttt{C++}}
\label{sec:poo}
\input{cpp-poo.tex}


\section{Le \texttt{C++} moderne}
\label{sec:cpp-moderne}
% -*- ispell-dictionary: "french" -*-

\begin{frame}[fragile]{Un peu d'histoire}
  \begin{description}
  \item[1972 :] première version de \texttt{C}
  \item[1983 :] première version de \texttt{C++} (alors \texttt{C with classes})
  \item[1989 :] premier standard pour \texttt{C} (\texttt{ANSI C})
  \item[1998 :] premier standard pour \texttt{C++}
  \item[2003 :] norme \texttt{C++03}
  \item[2011 :] norme \texttt{C++11} $\leftarrow$ début du \texttt{C++} moderne
  \item[2014 :] norme \texttt{C++14}
  \item[2017 :] norme \texttt{C++17}
  \end{description}
  \begin{lstlisting}[language=bash]
    --std=c++11
  \end{lstlisting}
\end{frame}

\begin{frame}[fragile]{De nouveaux tableaux}
  \begin{lstlisting}
std::array<int,5> a {};
// {} to initialize the contents of a    
a[1] = 2;
size_t a_size = a.size();

a[6] = 4; // undefined behaviour
a.at(6) = 4; // runtime error
  \end{lstlisting}

  Accès aux éléments :
  \begin{itemize}
  \item \texttt{\textit{array}[\textit{index}]}
  \item \texttt{\textit{array}.at(\textit{index})} vérifie l'indice
  \end{itemize}
\end{frame}

\begin{frame}[fragile]{De nouveaux tableaux}
  \begin{lstlisting}
std::vector<int> a;
a.push_back(10);
a[0];

std::vector<int> b (5);
b.pop_back();
size_t b_size = b.size(); // now 4
  \end{lstlisting}

  Accès aux éléments :
  \begin{itemize}
  \item \texttt{\textit{vector}[\textit{index}]}
  \item \texttt{\textit{vector}.at(\textit{index})} vérifie l'indice
  \end{itemize}
\end{frame}

\begin{frame}[fragile]{Une nouvelle manière de parcourir un conteneur}
   \begin{lstlisting}
std::vector<int> a;
// code
for (std::vector<int>::iterator it = a.begin() ; it != a.end() ; ++it) {
  // code
}
  \end{lstlisting}
  Parcours possible pour tous les itérables :
  \begin{itemize}
  \item \texttt{begin} et \texttt{end} sur l'objet
  \item \texttt{operator!=}, \texttt{operator++} et \texttt{operator*} sur l'itérateur
  \end{itemize}
  $\implies$ pas seulement les conteneurs séquentiels
\end{frame}

\begin{frame}[fragile]{Une nouvelle boucle \texttt{for}}
  \begin{lstlisting}
std::array<int,5> a;
// code
for (int &it : a) {
  // code
}
  \end{lstlisting}
  \texttt{range for} pour tous les itérables
\end{frame}

\begin{frame}[fragile]{Le type \texttt{auto}}
  \texttt{auto} : inférence de type  
  \begin{lstlisting}[escapechar=ù]
for (auto &iterator : ù\textit{iterable}ù) {
  // code
}
  \end{lstlisting}
\end{frame}

\begin{frame}[fragile]{Parlons performance}
  Que se passe-t-il si j'exécute le code ci-dessous ? 
  \begin{lstlisting}
std::string f () {
  // code
}
int main () {
  std::string x, y;
  // put some text in x and y
  std::string a(x);
  std::string b (x+y);
  std::string c (f());
  return 0;
}
  \end{lstlisting}
\end{frame}

\begin{frame}[fragile]{Parlons performance}
  Que se passe-t-il si j'exécute le code ci-dessous ?  
  \begin{lstlisting}
std::vector<std::string> f () {
  // return a large vector of long strings
}
int main () {
  std::vector<std::string> a (f());
  return 0;
}
  \end{lstlisting}
\end{frame}

\begin{frame}{Parlons performance}
  Pourquoi faire une copie ?
  $\implies$ Indiquer au compilateur que l'objet affecté est détruit immédiatement après !
\end{frame}

\begin{frame}[fragile]{\textit{Move semantics}}
  \begin{itemize}
  \item Possible depuis \texttt{C++11}
  \item Utilisation automatique
  \item Utilisation manuelle avec \texttt{std::move}
  \end{itemize}  
  \begin{lstlisting}[escapechar=ù]
ù\textit{new\_var}ù = std::move(ù\textit{old\_var}ù)
  \end{lstlisting}
\end{frame}

\begin{frame}[fragile]{\textit{Move semantics}}
  \begin{lstlisting}
class A {
  // stuff
public:
  ~A ();
  A (const A &);
  A (A &&); // move constructor
  A& operator= (const A &);
  A& operator= (A &&); // move assignment operator
  // more stuff
};
  \end{lstlisting}
\end{frame}

\begin{frame}{\textit{Move semantics}}
  Autant que possible, utiliser les méthodes par défaut !
\end{frame}

\begin{frame}[fragile]{\textit{Move semantics} en pratique}
  \begin{lstlisting}
class blob {
private:
  size_t m_size;
  char *m_data;
public:
  blob (): m_size(0), m_data(nullptr) {}
  ~blob () { delete[] m_data; }
};
  \end{lstlisting}
\end{frame}

\begin{frame}[fragile]{\textit{Move semantics} en pratique}
  \begin{lstlisting}
blob (const blob& other): m_size(other.m_size) {
  // copy constructor
  if (m_size) {
    m_data = new char[m_size];
    memcpy(m_data, other.m_data, m_size);
  } else {
    m_data = nullptr;
  }
}
  \end{lstlisting}
\end{frame}

\begin{frame}[fragile]{\textit{Move semantics} en pratique}
  Avec \textit{copy-and-swap} : 
  \begin{lstlisting}
friend void swap (blob& first, blob& second) {
  using std::swap;
  //
  swap(first.m_size, second.m_size);
  swap(first.m_data, second.m_data);
}

blob (blob&& other): blob() {
  // move constructor
  swap(*this, other);
}
  \end{lstlisting}
\end{frame}

\begin{frame}[fragile]{\textit{Move semantics} en pratique}
  \begin{lstlisting}
blob& operator= (blob other) {
  swap(*this, other);
  return *this;
}
  \end{lstlisting}
\end{frame}

\begin{frame}[fragile]{Retour vers les ressources}
  \begin{itemize}
  \item Ressource (encapsulée dans un objet) \og{}unique\fg{}
  \item Interdire la copie ?
  \end{itemize}
  \begin{lstlisting}
class A {
  // stuff
public:
  ~A ();
  A (const A &) = delete;
  A (A &&) = delete;
  A& operator= (const A) = delete;
};
  \end{lstlisting}
  \textit{delete} utilisable aussi avec l'héritage
\end{frame}

\begin{frame}[fragile]{Retour vers les ressources}
  Impossible d'utiliser une \textit{factory} ?
  \begin{lstlisting}
class A {
  // stuff
public:
  ~A ();
  A (const A &) = delete;
  A (A &&);
  A& operator= (const A);
};
  \end{lstlisting}
\end{frame}

\begin{frame}[fragile]{Retour vers les ressources}
  \begin{lstlisting}
int main () {
  A a = factory_for_A();
  A b, c;
  b = a; // Forbidden
  c = std::move(a); // a may end up unusable
  return 0;
}
  \end{lstlisting}
\end{frame}

\begin{frame}[fragile]{Les \textit{lambda}s}
  \begin{lstlisting}
void f (int i) {
  std::cout << i << ' ';
}

int main () {
  std::vector<int> vec;
  vec.push_back(10);
  // ...
  for_each(vec.begin(), vec.end(), f);
  std::cout << '\n';
  return 0;
}
  \end{lstlisting}
\end{frame}

\begin{frame}[fragile]{Un exemple de \textit{lambda}}
  \begin{lstlisting}
int main () {
  std::vector<int> vec;
  vec.push_back(10);
  // ...
  for_each(vec.begin(), vec.end(), [](int i) {
      std::cout << i << ' ';
    } );
  std::cout << '\n';
  return 0;
}
  \end{lstlisting}
\end{frame}

\begin{frame}[fragile]{{Les \textit{lambda}s}}
  \begin{lstlisting}
[](double x) -> double {
  if (x>0) {
    return 0.1*x;
  } else {
    return 0;
  }
}
  \end{lstlisting}
\end{frame}

\begin{frame}{{Les \textit{lambda}s}}
  Capture de variables :
  \begin{itemize}
  \item \texttt{[}\textit{var}\texttt{]} pour capturer par valeur
  \item \texttt{[\&}\textit{var}\texttt{]} pour capturer par référence
  \item \texttt{[=]} pour capturer toutes les variables par valeur
  \item \texttt{[\&]} pour capturer toutes les variables par référence
  \item \texttt{[=, \&}\textit{var}\texttt{]}
  \item \texttt{[\&, }\textit{var}\texttt{]}
  \end{itemize}
\end{frame}

\begin{frame}[fragile]{Des pointeurs intelligents}
  \begin{itemize}
  \item \texttt{shared\_ptr}
  \item \texttt{unique\_ptr}
  \end{itemize}
  \begin{lstlisting}
#include <iostream>
#include <memory>

int main () {
  std::shared_ptr<int> foo (new int(5));
  std::unique_ptr<int> bar (new int(3));
  std::cout << *foo << ' ' << *bar << '\n';
  return 0;
}
  \end{lstlisting}
\end{frame}

\begin{frame}[fragile]{Des pointeurs intelligents}
  Que se passe-t-il si j'exécute le code suivant ?
  \begin{lstlisting}
int main () {
  int *ptr = new int(5);
  std::shared_ptr<int> foo (ptr);
  std::cout << *foo << '\n';
  {
    std::shared_ptr<int> bar (ptr)
  }
  std::cout << *foo << '\n';
  return 0;
}
  \end{lstlisting}
\end{frame}

\begin{frame}{Des pointeurs intelligents}
  Ne pas initialiser de pointeur intelligent à partir d'un pointeur classique !
  
  Ou mieux, utiliser :
  \begin{itemize}
  \item \texttt{shared\_ptr} avec \texttt{make\_shared}
  \item \texttt{unique\_ptr} avec \texttt{make\_unique} (\texttt{C++14})
  \end{itemize}
\end{frame}

\begin{frame}[fragile]{Des pointeurs intelligents}
  \begin{lstlisting}
int main () {
  std::shared_ptr<int> foo = make_shared<int>(5);
  {
    std::shared_ptr<int> bar (foo);
  }
  std::unique_ptr<int> foobar = make_unique<double>(3.14);
  std::cout << *foo << ' ' << *foobar << '\n';
  return 0;
}
  \end{lstlisting}
\end{frame}



\section{Introduction à la librairie standard}
\label{sec:cpp-std}
\input{cpp-std.tex}


\section{Notions avancées}
\label{sec:cpp-avance}
% -*- ispell-dictionary: "french" -*-

\begin{frame}[fragile]{Avant d'aller plus loin}
  Pour activer les optimisations à la compilation :
  \begin{lstlisting}[language=bash]
g++ -Ox ...
  \end{lstlisting}

  \begin{itemize}
    \item 0 : aucune optimisation
    \item 1 à 3 : optimisation de plus en plus agressive de la vitesse
    \item \texttt{s} : optimisation de la taille de l'exécutable
  \end{itemize}
\end{frame}

\begin{frame}{Les macros}
  Mais qu'est réellement l'instruction \texttt{\#include <...>} ?
\end{frame}

\begin{frame}[fragile]{Les macros}
  \begin{itemize}
    \item Mécanisme d'exécution d'instructions à la compilation
    \item Hérité de C
  \end{itemize}

  \begin{lstlisting}
#include <cmath>

#define PI 3.1415
  \end{lstlisting}
\end{frame}

\begin{frame}[fragile]{Le mot clé \texttt{constexpr}}
  \texttt{constexpr} sert à indiquer qu'une expression est évaluable à la compilation.

  \begin{lstlisting}
constexpr double pi = 3.1415;
  \end{lstlisting}

  Il implique le qualificatif \texttt{const} pour les variables.  
\end{frame}

\begin{frame}[fragile]{Le mot clé \texttt{constexpr}}
  \begin{lstlisting}
constexpr std::vector<double> axis_x = {1, 0, 0};

constexpr int x = 1;
constexpr int y = 3;
constexpr int z = x + 2 * y;
  \end{lstlisting}
\end{frame}

\begin{frame}[fragile]{Le mot clé \texttt{constexpr}}
  \begin{lstlisting}
constexpr int factorial(int n) {
  return n <= 1 ? 1 : (n * factorial(n - 1));
}
  \end{lstlisting}
\end{frame}

\begin{frame}[fragile]{\texttt{constexpr} pour les classes}
  Il faut au moins un constructeur \texttt{constexpr}.

  \begin{lstlisting}
class GaussInt {
  public:
    constexpr GaussInt (int a=0, int b=0): m_real(a), m_imag(b) {}
    constexpr double norm () const { return std::sqrt(m_real*m_real + m_imag*m_imag); }
    constexpr GaussInt operator+(const GaussInt &other) const {
      return GaussInt(this->m_real + other.m_real, this->m_imag + other.m_imag);
    }
  };
  \end{lstlisting}
\end{frame}

\begin{frame}[fragile]{\texttt{constexpr} pour les classes}
  \begin{lstlisting}
constexpr GaussInt a;
constexpr GaussInt b (2,2);
constexpr GaussInt c (2,1);
constexpr GaussInt d (b + c);
constexpr double a_norm =  a.norm();
std::cout << "Norm of a: " << a_norm << std::endl;
std::cout << "Norm of d: " << d.norm() << std::endl;
constexpr auto fac = factorial(5);
std::cout << "Factorial of 5 is " << fac << std::endl;
  \end{lstlisting}
\end{frame}



\section*{Conclusion}
\label{sec:conclusion}
\begin{frame}{Ressources}
  Des ressources pour plus d'informations :
  \begin{itemize}
  \item \href{http://www.cplusplus.com/}{cplusplus.com}
  \item \href{https://en.cppreference.com/w/}{cppreference.com}
  \item \href{https://stackoverflow.com/}{Stack Overflow}
  \end{itemize}

  Une série d'articles intéressante : \href{https://ds9a.nl/articles/posts/cpp-intro/}{Modern C++ for C Programmers}
\end{frame}

\end{document}
